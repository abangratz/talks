\documentclass[style=paintings,display=slidesnotes,clock]{powerdot}
\pdsetup{
	palette=Charon
}
\usepackage[utf8]{inputenc}
\usepackage{courier}
\usepackage{xcolor}
\usepackage{listings}
\title{Bag o' Tricks: Unit Testing}
\author{Anton 'tony' Bangratz\\
	\url{https://abangratz.github.com/}\\
	\url{@tony_xpro}\\
\url{anton.bangratz@radarservices.com}}
\date{2012-01-18}

\begin{document}

\lstset{
	language=Ruby,
	commentstyle=\color{gray},
	keywordstyle=\color{yellow},
	stringstyle=\color{orange},
	basicstyle=\ttfamily\footnotesize\color{white},
}
\maketitle
\section{Introduction}
\begin{note}{About me}
	\begin{itemize}
		\item Ruby Developer
		\item XP Enthusiast
		\item Senior Developer at RadarServices
		\item Almost 2 centuries of experience; programming language agnostic, but like Ruby a lot. 
		\item Trying to convince people that automated Testing are a Good Thing since early 2000s
	\end{itemize}
\end{note}
\section{The Art of Testing}
\begin{note}{Testing}
	\begin{itemize}
		\item - Ask who is testing, automates, does TDD, CI, ...
		\item Tell about: Unit Testing, Integration testing
		\item Test Automation and CI/CD
	\end{itemize}
\end{note}

\section{Not (only) Rails}
\begin{note}{Not Rails}
	\begin{itemize}
		\item Ruby is more than Rails
		\item Testing should be done anyways
		\item Challenges: FS, IO, visuals
		\item Can use RSpec, MiniTest
		\item Examples use RSpec
		\item Rules:
			\begin{itemize}
				\item One Method, One Purpose
				\item One Test, One Assertion
			\end{itemize}
	\end{itemize}
\end{note}
\begin{slide}{One Method, One Purpose}
	\onslide*{1}{
		Consider:
		\begin{lstlisting}[frame=shadowbox]^^J
		class TitanClass^^J
		\ \ .^^J
		\ \ .^^J
		\ \ .^^J
		\ \ def doIt(*args)^^J
		\ \ \ \ args[:external_sources].each do |source|^^J
		\ \ \ \ \ \ args[:data_gatherers] do |gatherer| ^^J
		\ \ \ \ \ \ \ \ gatherer.doIt(source)^^J
		\ \ \ \ \ \ end^^J
		\ \ \ \ end^^J
		\ \ \ \ processed_data = []^^J
		\ \ \ \ current_gatherers.each do |gatherer|^^J
		\ \ \ \ \ \ processed_data << process_data(gatherer)^^J
		\ \ \ \ end^^J
		\ \ \ \ reports = \{\}^^J
		\ \ \ \ processed_data.each do |data|^^J
		\ \ \ \ \ \ reports[data.id] = Report.new(data)^^J
		\ \ \ \ end^^J
		\ \ \ \ ReportPrinter.print_all(reports)^^J
		\ \ end^^J
		end^^J
		\end{lstlisting}
	This is a contrived example. The one I modelled that from is worse. By far.
	}
	\onslide*{2}{
		Test:
		\begin{lstlisting}[frame=shadowbox]^^J
		require 'spec_helper'^^J
		^^J
		describe TitanClass do^^J
		\ \ it "should do everything" do^^J
		\ \ \ \ source = double(Source)^^J
		\ \ \ \ gatherer = double(Gatherer)^^J
		\ \ \ \ args = \{external_sources: [source],^^J
		\ \ \ \ \ \ data_gatherers: [gatherer]\}^^J
		\ \ \ \ processed = double("ProcessedData")^^J
		\ \ \ \ subject.should_receive(process_data)^^J
		\ \ \ \ \ \ .with(gatherer).and_return(processed)^^J
		\ \ \ \ report = double(Report)^^J
		\ \ \ \ Report.should_receive(:new).with(processed).^^J
		\ \ \ \ \ \ and_return(report)^^J
		\ \ \ \ ReportPrinter.should_receive(:print_all).with([report])^^J
		\ \ \ \ subject.doIt(args)^^J
		\ \ end^^J
		end^^J
		\end{lstlisting}
	}
	\onslide*{3}{
		Test:
		\begin{lstlisting}[frame=shadowbox]^^J
		require 'spec_helper'^^J
		^^J
		describe TitanClass do^^J
		\ \ it "should do everything" do^^J
		\ \ \ \ source = double(Source)^^J
		\ \ \ \ gatherer = double(Gatherer)^^J
		\ \ \ \ args = \{external_sources: [source],^^J
		\ \ \ \ \ \ data_gatherers: [gatherer]\}^^J
		\ \ \ \ processed = double("ProcessedData")^^J
		\ \ \ \ subject.should_receive(process_data)^^J
		\ \ \ \ \ \ .with(gatherer).and_return(processed)^^J
		\ \ \ \ # Yikes, a stub on the subject under test!^^J
		\ \ \ \ report = double(Report)^^J
		\ \ \ \ Report.should_receive(:new).with(processed).^^J
		\ \ \ \ \ \ and_return(report)^^J
		\ \ \ \ # Stubbing ".new" is rarely a good idea^^J
		\ \ \ \ ReportPrinter.should_receive(:print_all).with([report])^^J
		\ \ \ \ # Huh. We know that a ReportPrinter prints reports,^^J
		\ \ \ \ # but that's more than we actually should know ...^^J
		\ \ \ \ subject.doIt(args)^^J
		\ \ end^^J
		end^^J
		\end{lstlisting}
		Let's clean up this mess, step by step ...
	}

\end{slide}
\section{One Test, One Assertion}
\begin{slide}{Example}
	\begin{lstlisting}[frame=shadowbox]^^J
	describe SmartClass do^^J
	\ \ it "invokes the data gatherer" do^^J
	\ \ \ \ data_gatherer = double(DataGatherer)^^J
	\ \ \ \ source = double(Source)^^J
	\ \ \ \ data_gatherer.should_receive(gather).with(source))^^J
	\ \ \ \ # Our only assertion!^^J
	\ \ \ \ subject.data_gatherer = data_gatherer # DI for the win^^J
	\ \ \ \ subject.gather_from_source(source)^^J
	\ \ end
	end^^J
	...^^J
	class SmartClass^^J
	^^J
	\ \ attr_accessor :data_gatherer^^J
	^^J
	\ \ def gather_from_source(source)^^J
	\ \ \ \ data_gatherer_gather(source)^^J
	\ \ end^^J
	end^^J
	\end{lstlisting}
\end{slide}
\section{Mocking and Stubbing}
\begin{note}{Mocking, Stubbing}
	\begin{itemize}
		\item Mocks are not stubs
		\item Use them instead of doing expensive things
		\item Use whenever not referring to the SUT
		\item NEVER use when referring to the SUT
	\end{itemize}
\end{note}
\section{Testing Modules}
\begin{slide}{Example}
	\begin{lstlisting}[frame=shadowbox]^^J
	class TestSmartModule^^J
	\ \ include SmartModule^^J
	end^^J
	describe SmartModule do^^J
	\ \ subject \{ TestSmartModule.new \}^^J
	\ \ it "does something" do^^J
	\ \ \ \ subject.something.should ...^^J
	\ \ end^^J
	end^^J
	\end{lstlisting}
\end{slide}
\begin{note}{Testing Modules}
	\begin{itemize}
		\item Easy: make a dummy class and include the module
		\item USE subject!
		\item CAVEAT: don't use 'DummyClass' everywhere - there will be constant bleeding
	\end{itemize}
\end{note}
\section{Testing IO}
\begin{note}{Testing Modules}
	\begin{itemize}
		\item Mention MockFS, FakeFS
		\item Wrapper and/or stubbing better.
		\item DI even better.
	\end{itemize}
\end{note}
\section{Well, Rails}
\section{Anonymous Controllers}
\begin{note}{Anonymous Controllers}
	Test functionality for all or many controllers \\
	that has been added via inheritance (e.g. ApplicationController)
\end{note}
\begin{slide}{Example}
	\begin{lstlisting}[frame=shadowbox]^^J
	describe SomeClass, type: 'controller' do^^J
	\ \ controller(ApplicationControllerOrSubClass) do^^J
	\ \ \ \ def index^^J
	\ \ \ \ \ \ ... # do something^^J
	\ \ \ \ end^^J
	\ \ end^^J
	\ \ ^^J
	\ \ it "makes something happen" do^^J
	\ \ \ \ ...  # assert something happens^^J
	\ \ \ \ get :index^^J
	\ \ end^^J
	end^^J
	\end{lstlisting}
\end{slide}
\section{Testing Models?}
\begin{note}{Testing Models}
	\begin{itemize}
		\item Make sure that you don't use Fixtures or Real Database\texttrademark\  access
		\item Don't test for properties/columns, maybe validations and transformation methods, 
		\item but ... see next
	\end{itemize}
\end{note}
\section{Another Layer}
\begin{note}{Another Layer}
	\begin{itemize}
		\item Omakase vs. Prime (thanks @steveklabnik)
		\item Draper, Objects On Rails, ...
	\end{itemize}
\end{note}
\section{TDD}
\begin{note}{TDD}
	\begin{itemize}
		\item It's hard.
		\item It's an art.
		\item It's an acquired taste.
		\item Leads to better design decisions
		\item Keeps coverage high (enough)
		\item Makes sure tests are fast
		\item can be done multi-layered
	\end{itemize}
\end{note}
\section{ Frameworks }
\section{ MiniTest or RSpec? }
\section{ That's all, folks. }
\begin{slide}{Thanks}
	\begin{itemize}
		\item Stephan Kämper
		\item Lisa Crispin
		\item Elisabeth Hendricks
		\item Michael Bolton
		\item Avdi Grimm
		\item Steve Klabnik
		\item Jeremy Evans
		\item Martin Gamsjäger
		\item Robert C. Martin (Uncle Bob)
		\item Andreas Kopecky
		\item Jürgen Strobel
		\item Andreas Tiefenthaler
		\item Michael Kohl
		\item Floor Drees
		\item Friends and colleagues that inspired me
	\end{itemize}
\end{slide}
\end{document}
