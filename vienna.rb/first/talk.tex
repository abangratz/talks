\documentclass[style=husky,display=slides,clock]{powerdot}
\usepackage{courier}
\usepackage{xcolor}
\usepackage{listings}
\title{Bag o' Tricks: Unit Testing}
\author{Anton Bangratz\\
	\url{https://abangratz.github.com/}\\
	\url{@tony_xpro}\\
\url{anton.bangratz@radarservices.com}}
\date{2012-01-18}

\begin{document}

\lstset{
	language=Ruby,
	identifierstyle=\color{red},
	commentstyle=\color{gray},
	keywordstyle=\color{blue},
	stringstyle=\color{orange},
	basicstyle=\ttfamily\tiny\color{black},
}
\maketitle
\section{Frameworks}
\begin{slide}{Minitest}
	\onslide*{1}{
		\begin{itemize}
			\item
		\end{itemize}
	}
	\onslide*{2}{
		\begin{lstlisting}[frame=shadowbox]^^J
		class Test < Object^^J
		\ \ def X(y)^^J
		\ \ end^^J
		end^^J
		puts "Hello World"^^J
	\end{lstlisting}
	}
\end{slide}
\begin{slide}{RSpec}
\end{slide}
\begin{slide}{Mix and Match}
	%http://myronmars.to/n/dev-blog/2012/07/mixing-and-matching-parts-of-rspec
\end{slide}
\section{Not (only) Rails}
\begin{slide}{One Method, One Purpose}
\end{slide}
\begin{slide}{One Test, One Assertion}
\end{slide}
\begin{slide}{Mocking and Stubbing}
\end{slide}
\begin{slide}{Testing Modules}
\end{slide}
\begin{slide}{Testing IO}
\end{slide}
\section{Well, Rails}
\begin{slide}{Anonymous Controllers}
\end{slide}
\begin{slide}{Testing Models?}
\end{slide}
\begin{slide}{Another Layer}
\end{slide}
\section{TDD}
\begin{slide}{How To?}
\end{slide}
\begin{slide}{Advantages}
\end{slide}
\end{document}
